
\newglossaryentry{gravitational wave}{
  name = {gravitational wave},
  description = {A gravitational wave is a propogating perturbation of spacetime.}
}

\newglossaryentry{GW150914}{
  name = {GW150914},
  description = {The first gravitational wave event to be observed. The observation was made by the two \gls{LIGO} detectors in the United States of America, located at Hanford, Washington; and Livingston, Louisiana}
}

\newglossaryentry{VIRGO}{
  name = {VIRGO},
  description = {
    A gravitational wave detector in Italy.
  }
}

\newglossaryentry{LIGO}{
  name = LIGO,
  description = {
    The Laser Interferometer Gravitational-wave Observatory is a
    joint project of California Institute of Technology (CalTech) and
    the Massechusets Institute of Technology (MIT), in which laser
    interferometers of a similar design to the Michelson
    interferometer, famous for its use in disproving the existence of
    the ``luminous aether'', are used to detect small length
    perturbations over distances of 4km. }
}

\newglossaryentry{DetChar}{
  name = detector characterisation,
  description = {DetChar, or detector characterisation, is the process of analysing the noise sources and calibration of the detector, as well as identifying ``glitches'', transient noise events which can interfere with burst searches, and ``lines'', sources of noise which exist in a narrow frequency band which can interfere with long-integration time searches.}
}

\newglossaryentry{burst}{
  name = burst,
  description = {A burst is a short-lived, transient gravitational wave event.}
}

\newglossaryentry{chirp mass}{
  name = {chirp mass, $\mathcal{M}$},
  description = {
    The chirp mass of a binary system is defined as 
    \[ \mathcal{M} = \frac{ (m_1 m_2)^{3/5} }{(m_1 + m_2)^{1/5} } \]
    for $m_1$ and $m_2$ the masses of the two components of the binary,
    and determines the amplitude and the frequency evolution of a
    chirp from a coalescence%
  }}

\newglossaryentry{SNR}{
  name = {signal-to-noise ratio, SNR},
  description = {
    The ratio of signal power to noise power in a given signal. For a
    gravitational wave detector it is defined
    \[ \rho^2 = \int_0^\infty \frac{ 4 \left| h(f) \right|^2
    }{S_n(f)} \dd{f} \] for $h(f)$ the strain of the signal in the frequency
    domain, and $S_n$ is the power spectral density of the detector%
  }
}

\newglossaryentry{characteristic strain}{
  name = characteristic strain,
  description = {
    The characteristic strain is designed to take into account the change of a signal's frequency when it is integrated. It is defined
    \[h^2~c(f) = 4 f^2 \abs{h(f)}^2 \] for $h(f)$ the strain of the
    signal in the frequency domain, and $f$ the frequency%
  } }

\newglossaryentry{finesse}{
  name = finesse,
  description = {
The number of times a beam reflects between ends of a Fabry-Perot interferometer.
  } }


%%% Local Variables: 
%%% mode: latex
%%% TeX-master: t
%%% End: 
