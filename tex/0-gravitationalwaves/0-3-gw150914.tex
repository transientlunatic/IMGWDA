The first detection of gravitational waves was announced by the LIGO
Scientific Collaboration and the VIRGO Collaboration on 11 February
2016\cite{2016PhRvL.116f1102A,2016arXiv160203843T,2016arXiv160203839T}. The
event observed was determined to have been a binary black hole
coalescence between a 29 and a 36 solar mass black hole, at a distance
of $\SI{410}{\mega pc}$\cite{2016arXiv160203840T}. While this
observation demonstrated the existence of binary black hole
systems\cite{2016ApJ...818L..22A}, it is somewhat problematic, as the
masses of the two components of the binary do not fit theoretical
predictions of binary black hole formation, and this has lead to a
number of theoretical explanations for the system's evolution in the
months following the detection, including the potential that black
holes no not receive large natal kicks\cite{2016arXiv160204531B}, or
that the black holes were primordial\cite{2016arXiv160308338S}, or
were dark matter halo objects\cite{2016arXiv160300464B}. The
observation has also provided extra evidence for the rate of black
hole mergers in the local universe\cite{2016arXiv160203842A}.

The LIGO Scientific Collaboration also shared data with a number of
partner observatories in relatively low-latency of the detection,
allowing a programme of
electromagnetic\cite{2016arXiv160208492A,2016arXiv160407864A} and
neutrino\cite{2016arXiv160205411A} observations to be initiated. 25 of
the 63 observing teams who were advised of the detection through
private Gamma-ray Network Circulars, only one reported evidence of an
electromagnetic counterpart to the event, the Fermi GBM observing
team\cite{2016arXiv160203920C}, who reported a hard X-ray event
lasting around 1 second with a sky location consistent with the sky
localisation of the gravitational wave event. While the Fermi event
was not observed by other teams working at similar
wavelengths\cite{2016ApJ...820L..36S,2016arXiv160204488F,2016MNRAS.tmpL..45E}
it precipitated a flurry of papers which investigated the
astrophysical situations which might lead to an electromagnetic
counterpart of a black hole
coalescence\cite{2016ApJ...819L..21L,2016arXiv160407132J}.

%%% Local Variables: 
%%% mode: latex
%%% TeX-master: "../../document"
%%% End: 
