
\subsection{Shot noise}
\label{sec:shot-noise}

\subsection{Radiation pressure noise}
\label{sec:radi-press-noise}

\subsection{Thermal noise---Coatings}
\label{sec:therm-noise-coat}

\subsection{Thermal noise---suspensions}
\label{sec:therm-noise-susp}


\subsection{Newtonian Noise}
\label{sec:newtonian-noise}

Newtonian noise, or gravitational gradient noise, is the strain
produced by gravitational coupling between local mass density
variations and the test masses in the interferometer. Examples of
significant sources of Newtonian noise include clouds passing overhead
the detector, and seismic perturbations in the local ground density.

\subsection{Seismic Noise}
\label{sec:seismic-noise}

Seismic noise is the result of strain introduced into the
interferometer through movement of the ground, which can be the result
of geophysical activity, tidal activity, or anthropogenic sources of
seismic noise, such as road traffic or railways. Seismic noise is also
a source of Newtonian noise (see section \ref{sec:newtonian-noise})
due to density fluctuations as the seismic wave passes through the
ground.

% \marginpar{
%   \begin{tabular}{ccl}
%     $f$ [Hz] & $D$ [km] & Sources \\
%     0.01--1.0    &  1000         & earthquakes, microseism
%   \end{tabular}
% }

Seismic noise limits the sensitivity of the second generation
detectors at low frequencies ($f < \SI{10}{\hertz}$), but it is
present as a noise source across the passband of the detector. The
seismic noise shows a pair of notable peaks below the $\SI{1}{\hertz}$
level, one caused by ocean swell, which has a period around 4 to 30
seconds, and a second caused by standing seismic modes in the Earth
which spans the range of 30 to 1000 seconds.

Seismic isolation is used in detectors to reduce the noise level due
to seismic activity. This takes two forms: active isolation, and
passive isolation. The former is accomplished by mounting optical
components on servo-controlled systems which are controlled, via a
feedback-loop, to a seismometer. The latter is reduced by suspending
the optics as a component in a pendulum system. In the Advanced LIGO
design this involves the test masses and their associated mirrors
composing the final component in a quadruple pendulum suspension.



\subsection{Other noise sources}
\label{sec:other-noise-sources}
There are numerous additional noise sources within the interferometer.

%%% Local Variables: 
%%% mode: latex
%%% TeX-master: "../../document"
%%% End: 
