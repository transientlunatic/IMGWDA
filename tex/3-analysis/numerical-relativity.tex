In order to produce templates for matched filtering an understanding
of the shape of the signal which is expected is required. For
gravitational wave events these signals usually have a complicated
form, and are analytically intractable, leading to the requirement
that these be calculated numerically. This has become possible in the
last decade or so, leading to the production of a number of catalogues
of
waveforms\cite{PhysRevLett.106.241101,gatechcat,2016PhRvD..93d4006H}. These
waveforms are computationally expensive to produce, and so ``post-Newtonian
approximants'' to them have been produced, following two separate
approaches, the ``Phenom'' classes of
approximant\cite{2016PhRvD..93d4007K,2007CQGra..24S.689A} and
``effective one-body'' classes\cite{2007PhRvD..76j4049B}.


%%% Local Variables: 
%%% mode: latex
%%% TeX-master: "../../document"
%%% End: 