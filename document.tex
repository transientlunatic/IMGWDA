\documentclass{kentigern}
\usepackage[utf8]{inputenc}
\makeatletter
\input{journal-macros}
\makeatother
\usepackage{lipsum}
\usepackage[acronym,toc,nopostdot,xindy,style=index]{glossaries} %previously index
\usepackage{siunitx}

\usepackage{amsmath,amsfonts,amssymb}
\usepackage{tensor}
\makeglossaries
\usepackage{tabularx}
\usepackage{type1cm}
\usepackage{lettrine}
%\usepackage{physicsplus}
\usepackage{caption}
\usepackage{pgfplots}
\usepackage{rotating}
%\usepackage{pgfgantt}
\usepackage{environ}
\usepackage{setspace}

\usepackage{pdflscape}

\usetikzlibrary{bayesnet}

\usepackage{doi}

\usepackage[
  backend=biber,
  sorting=none,
  style=numeric,
]{biblatex}
\addbibresource{bibliography/observing-runs.bib}
\addbibresource{bibliography/general-relativity.bib}
%\addbibresource{bibliography/books.bib}
\addbibresource{bibliography/mypapers.bib}
\addbibresource{bibliography/sources.bib}
\addbibresource{bibliography/relativity.bib}
\addbibresource{bibliography/detectors.bib}
\addbibresource{bibliography/events.bib}
\addbibresource{bibliography/probability.bib}
\addbibresource{bibliography/data-analysis.bib}
\addbibresource{bibliography/sources.bib}
\addbibresource{bibliography/gaussianprocess.bib}
\addbibresource{bibliography/gammarayburst.bib}
%

\linespread{1.5}

\setsecnumdepth{subsubsection}
\settocdepth{subsection}

\title{Plausible inference methods for gravitational wave data analysis}
\author{Daniel Williams}

%% This file is intended to contain all of the macros defining the
%% various mathematical and physical quantities used in this document in
%% order to maintain notational consistency throughout the document and
%% its glossary.

% Margin notes

\newcommand{\marginnote}[1]{
        \refstepcounter{footnote}
\footnotemark\marginpar{\footnotemark}\footnotetext{#1}}


\NewEnviron{sidefigure}[2]{
  \sidebar{
    \BODY
    \captionof{figure}{#1 \label{#2}}
  }
}

\providecommand{\msolar}{\mathrm{M{_\odot}}}
\providecommand{\mat}[1]{\mathsf{#1}}
\providecommand{\ex}{\mathbb{E}\,}

% New Operators
\DeclareMathOperator{\vary}{var}
\DeclareMathOperator{\cov}{cov}
\DeclareMathOperator{\real}{Re}
\providecommand{\rmse}{\mathrm{RMSE}}

% Calculus operators
\providecommand{\dd}{\,\mathrm{d}}
\providecommand{\dalembert}{\square\,}
\providecommand{\ii}{\text{i}}

% Linear algebra operators
\providecommand{\transpose}{\intercal}
\providecommand{\tt}{\transpose}
\providecommand{\ten}[1]{\mathsf{#1}}

% Software Packages
\providecommand{\imrphenomp}{\texttt{IMRPhenomP}}
\providecommand{\lalsim}{\texttt{LALSimulation}}

% New Units
\providecommand{\solMass}{\ensuremath{\mathrm{M}_{\odot}}}
%\DeclareSIUnit\parsec{pc}

% Galactic astronomy
\providecommand{\numberGalaxies}{N_\mathrm{G}}

% Gravitational waves : tensors
\providecommand{\barh}{\bar{h}\indices{_{\mu\nu}}}

% Gravitational wave detectors
\providecommand{\horizonDistance}{\ensuremath\mathcal{D}_{\mathrm{hor}}}
\providecommand{\GP}{\gls{gp}}

% Pipelines
\providecommand{\olib}{\texttt{oLIB}}
\providecommand{\cwb}{\texttt{cWB}}
\providecommand{\bayeswave}{\texttt{Bayeswave}}
\providecommand{\minke}{\texttt{Minke}}

% Software
\providecommand{\lalsuite}{\texttt{LALSuite}}
\providecommand{\lalsimulation}{\texttt{LALSimulation}}
\providecommand{\imrp}{\texttt{IMRPhenom\,v2}}
\providecommand{\seobnr}{\texttt{SEOBNR}}
\providecommand{\heron}{\texttt{heron}}
% latin
\providecommand{\map}{maximum \emph{a posteriori}}

% Gaussian processes
\providecommand{\set}[1]{\mathcal{#1}}
\providecommand{\gp}{\mathcal{G\!P}}
\providecommand{\GP}{Gaussian Process\renewcommand{\GP}{GP}}
\providecommand{\NR}{Numerical Relativity\renewcommand{\NR}{NR}}
\providecommand{\PE}{parameter estimation\renewcommand{\PE}{PE}}
\providecommand{\GW}{gravitational wave\renewcommand{\GW}{GW}}
\providecommand{\EI}{\mathbb{E} \mathrm{I}}

\providecommand{\trainingpoints}{\mathcal{X}}
\providecommand{\trainingobservations}{\mathcal{Y}}
\providecommand{\trainingdata}{(\trainingpoints, \trainingobservations)}

\providecommand{\kernel}[1]{\mathsf{#1}}
\providecommand{\SE}{\kernel{SE}}
\providecommand{\Con}{\kernel{C}}
\providecommand{\Lin}{\kernel{Lin}}
\providecommand{\Per}{\kernel{Per}}
\providecommand{\RQ}{\kernel{RQ}}
\providecommand{\Mat}{\kernel{M52}}

\providecommand{\numbertrainingpoints}{$12,325$}
\providecommand{\numbertrainingwaveforms}{no. Waveforms}

\newcommand{\subgw}{_{\textrm{\scriptsize{GW}}}}
\newcommand{\ee}[1]{\ensuremath{\!\times\!10^{#1}}}
\newcommand{\prob}{{\rm Pr}}
\newcommand{\grbrate}{{{\mathcal R}_{\mathrm{grb}}}}
\newcommand{\cbcrate}{{{\mathcal R}}}
\newcommand{\diff}{{\mathrm d}}
\newcommand{\rhostar}{{\rho^*}}
\newcommand{\dhor}{\ensuremath{{\mathcal D}_{\mathrm{hor}}}}
\newcommand{\dinsp}{\ensuremath{{\mathcal D}_{\mathrm{insp}}}}
\newcommand{\latin}[1]{\textit{#1}}
\newcommand{\mpc}{\mathrm{Mpc}}
\newcommand{\yr}{\mathrm{yr}}

\theoremstyle{definition}
\newtheorem{definition}{Definition}[section]
\newtheorem{theorem}{Theorem}[section]
\newtheorem{corollary}{Corollary}[theorem]
\newtheorem{lemma}[theorem]{Lemma}
% allows for temporary adjustment of side margins
%\usepackage{chngpage}


\usetikzlibrary{bayesnet}


% The glossary
\loadglsentries[main]{chapters/glossary/glossary}
\makeglossaries

\maxsecnumdepth{subsubsection}

\newcommand{\thesistitle}{%
  \thispagestyle{empty}
  %\begingroup%
  \hbox{%
    \hspace*{-0.2\textwidth}%
    \parbox[b]{1.25\textwidth}{%
      \vbox{%
        \begin{center}
          {\noindent\HUGE\bfseries%
            {Inference Methods\\\textit{for}\\[.7cm] Gravitational Wave Data Analysis}}\\[2\baselineskip]
        \end{center}
        \begin{center}
          {\noindent\Large\itshape{On the application of Bayesian inference and modern modelling methods to astrophysical problems in the era of gravitational wave observation.}}\\[2\baselineskip]
        \end{center}
        \vspace*{3\baselineskip}
        {\Large Daniel Williams}\\
        %{\Large MSci}\\[3\baselineskip]
        {2019}
      }
    }
    }
    \vfill
    \hspace*{-0.2\textwidth}%
    \parbox[b]{1.25\textwidth}{%
      \vbox{%
        {Submitted in fulfilment of the requirements for the degree of Doctor of Philosophy.}\\
        {School of Physics \& Astronomy}\\
        {College of Science and Engineering}\\
        {University of Glasgow $\cdot$ Oilthigh Ghlaschu}
      }%
    }%
  \null%
  %\endgroup%
  \newpage
}%



\begin{document}
\openright
\frontmatter
\thesistitle
\newpage \newpage

\begin{abstract}
  Einstein's publication of the \textit{general theory of relativity} in 1915, and the discovery of a wave-like solution to the field-equations of that theory sparked a century-long quest to detect \textit{gravitational waves},
  the illusive metric disturbances which were predicted to ripple-away from some of the most energetic events in the universe, such as supernovae and colliding black holes.
  While this quest was completed in September 2015, with the Laser Interfermeter Gravitational-wave Observatory (LIGO) observation of a gravitational wave produced by a pair of coalescing black holes,
  the age of gravitational wave detection has by no means come to an end,
  with the prospect of myriad detections in the near future to analyse.
\end{abstract}
\newpage

Copyright 2019 Daniel Williams, all rights reserved.\\

This document was built with version 3rdyear-23-g9a21f3f-*
 of the git repository, and has been assigned the identifier LIGO-P1900001 by the LIGO Document Control Centre.

\newpage
\tableofcontents
\newpage
\listoffigures
\newpage
\listoftables
\newpage

\printglossary[type=\acronymtype]


%If I've learned nothing else in my time working in the Institute for Gravitational Research at the University of Glasgow, 
and living in a city just a short drive south of the Scottish Highlands, 
it is that completing a PhD and climbing a mountain have their similarities.

While at times the weather is fine, and you have a clear view of the task before you,
more often than not you find yourself walking through low cloud, and trying to pick out a 
well-concealed path, or searching for that same path after you've strayed away from it.

The journey is tiring, and there are times that it seems like it would just be easier to turn around, and go back to the flat,
but the promise of the fine view at the top, and the achievement of reaching it are a strong motivation.

While there are times that one finds oneself climbing alone, it is a rare expedition where you are completely alone.
There are those who show us the way, and in the last few years many people have helped me gain my bearings; 
\ldots{}

There are others who set off on a journey with you, but take different paths, making their own way to the top; 
without the perspective of my fellow postgraduate students my experience here would have been undoubtedly poorer,
and my chances of success certainly slimmer.

% \part{Outline \& Review of Gravitational Wave Literature}
% \label{part:introduction}
\newpage
\section{Notational conventions}
\label{sec:notation-conventions}

Throughout this work I take the convention that the metric tensor, $\ten{g}$ should be positive, having the signature $(-,+,+,+)$, and likewise the Riemann, $\ten{R}$, and Einstein, $\ten{E}$ tensors should also be positive, following the ``spacelike convention'' of Landau \& Lifshitz, and the convention of Misner, Thorne, and Wheeler (1973). I also adopt the convention of using greek indices for four-dimensional tensor quantities, such as 4-vectors, and latin indices otherwise. The Einstein summation convention is also assumed throughout for repeated indices. The reader should note that while the discussion of metrics in the context of general relativity is limited to four-dimensions, those metrics used in feature-space descriptions of data, especially in the context of Gaussian process regression, are not. 

\mainmatter

\chapter[Gravitational Waves: Generation, propagation, and detection]{Gravitational Waves: generation, propagation, and detection}
\label{chapter:intro}

\chapterprecis{
  This chapter introduces gravitational waves, and briefly discusses their description in general relativity, and their propagation through spacetime.
  The chapter then continues to discuss methods by which they might be detected, and a brief history of attempts to do so.

  Section \ref{sec:gw} contains an introduction and overview of the radiation predicted by general relativity: gravitational waves, and gives a very concise introduction to how these waves propagate.

  Section \ref{sec:gw:strain} briefly summarises a number of conventions which are used within gravitational wave astronomy, and this work, for describing the strain effect of gravitational waves, and related quantities.

  Section \ref{sec:detectors} gives an overview of gravitational wave detection, including the early attempts to identify measurable quantities from the theory, to the operational constraints of modern detectors.

  The vast majority of material in this chapter is review material, however a number of sensitivity curve plots were produced by a novel Python package, \texttt{gravpy}, which is described in appendix \ref{app:gravpy}.
}

\input{chapters/introduction/gravitational-waves}
% detectors
%\input{chapters/introduction/detectors}

\glsresetall % reset all of the acronyms so they show in full
\chapter{Astrophysical sources of gravitational waves and their waveforms}
\label{cha:sources}
\input{chapters/sources/sources}
\glsresetall % reset all of the acronyms so they show in full


% \part{Data Analysis for Gravitational Wave Detectors}
% \label{part:data-analysis}
\glsresetall % reset all of the acronyms so they show in full

\chapter{Bayesian inference}
\label{cha:bayesian-inference}
\input{chapters/analysis/probability}
\glsresetall % reset all of the acronyms so they show in full

\chapter{Hierarchical Modelling of Gamma Ray Bursts}
\label{cha:gamma-ray-burst}
---
title: Gamma ray burst
abbreviation: GRB
---

Gamma ray burst. These are short-lived electromagnetic events which are highly luminous, especially within the gamma ray regime of the spectrum. Events lasting for less than around \si{2}{\second} are classified as ``short'' GRBs (sGRB), while the rest are long GRBs. sGRBs are believed to be the result of binary neutron star coalescence.
\glsresetall % reset all of the acronyms so they show in full

\chapter{Gaussian processes for surrogate modelling}
\label{cha:gaussian-process}
\input{chapters/analysis/gaussian-processes}
\glsresetall % reset all of the acronyms so they show in full

\chapter{Heron: A Gaussian process regression approach to modelling gravitational waveforms}
\label{cha:heron}
\input{chapters/heron/heron}
\input{chapters/heron/paper}


\appendices
\chapter{Gravpy}
\label{app:gravpy}


% % The glossary
% % \glsaddall
\printglossary

\medskip
\printbibliography[title={Bibliography}]

\end{document}
